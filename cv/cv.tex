%%%%%%%%%%%%%%%%%%%%%%%%%%%%%%%%%%%%%%%%%
% "ModernCV" CV and Cover Letter
% LaTeX Template
% Version 1.3 (29/10/16)
%
% This template has been downloaded from:
% http://www.LaTeXTemplates.com
%
% Original author:
% Xavier Danaux (xdanaux@gmail.com) with modifications by:
% Vel (vel@latextemplates.com)
%
% License:
% CC BY-NC-SA 3.0 (http://creativecommons.org/licenses/by-nc-sa/3.0/)
%
% Important note:
% This template requires the moderncv.cls and .sty files to be in the same 
% directory as this .tex file. These files provide the resume style and themes 
% used for structuring the document.
%
%%%%%%%%%%%%%%%%%%%%%%%%%%%%%%%%%%%%%%%%%

%----------------------------------------------------------------------------------------
%	PACKAGES AND OTHER DOCUMENT CONFIGURATIONS
%----------------------------------------------------------------------------------------

\documentclass[11pt,a4paper,sans]{moderncv} % Font sizes: 10, 11, or 12; paper sizes: a4paper, letterpaper, a5paper, legalpaper, executivepaper or landscape; font families: sans or roman

\moderncvstyle{classic} % CV theme - options include: 'casual' (default), 'classic', 'oldstyle' and 'banking'
\moderncvcolor{blue} % CV color - options include: 'blue' (default), 'orange', 'green', 'red', 'purple', 'grey' and 'black'

\usepackage{lipsum} % Used for inserting dummy 'Lorem ipsum' text into the template

\usepackage[scale=0.8]{geometry} % Reduce document margins
\setlength{\hintscolumnwidth}{1.9cm} % Uncomment to change the width of the dates column
%\setlength{\makecvtitlenamewidth}{10cm} % For the 'classic' style, uncomment to adjust the width of the space allocated to your name

%----------------------------------------------------------------------------------------
%	NAME AND CONTACT INFORMATION SECTION
%----------------------------------------------------------------------------------------

\firstname{Ivan} % Your first name
\familyname{Yakovlev} % Your last name

% All information in this block is optional, comment out any lines you don't need
\title{Curriculum Vitae}
%\address{123 Broadway}{City, State 12345}
%\mobile{(000) 111 1111}
%\phone{(000) 111 1112}
%\fax{(000) 111 1113}
\email{ivan.yakovlev@u-bordeaux.fr}
\homepage{iyakovlev23.github.io}{iyakovlev23.github.io} % The first argument is the url for the clickable link, the second argument is the url displayed in the template - this allows special characters to be displayed such as the tilde in this example
%\extrainfo{additional information}
%\photo[80pt][0pt]{3x4} % The first bracket is the picture height, the second is the thickness of the frame around the picture (0pt for no frame)
%\quote{"A witty and playful quotation" - John Smith}

%----------------------------------------------------------------------------------------

\begin{document}

%----------------------------------------------------------------------------------------
%	COVER LETTER
%----------------------------------------------------------------------------------------

% To remove the cover letter, comment out this entire block

%\clearpage

%\recipient{HR Department}{Corporation\\123 Pleasant Lane\\12345 City, State} % Letter recipient
%\date{\today} % Letter date
%\opening{Dear Sir or Madam,} % Opening greeting
%\closing{Sincerely yours,} % Closing phrase
%\enclosure[Attached]{curriculum vit\ae{}} % List of enclosed documents

%\makelettertitle % Print letter title

%\lipsum[1-2] % Dummy text
%\lipsum[4] % Dummy text

%\makeletterclosing % Print letter signature

%\newpage

%----------------------------------------------------------------------------------------
%	CURRICULUM VITAE
%----------------------------------------------------------------------------------------

%{\Huge Ivan Yakovlev -- {Cover letter}}
%\vspace{4mm}

%Dear organizers of the Research School,
%\vspace{4mm}

%My research topic deals with enumeration of square-tiled surfaces and, more generally,  combinatorial maps, and has its roots in the problem of computation of Masur-Veech volumes of moduli spaces of translation surfaces.
%\vspace{4mm}

%Moreover, I use computer experimentation on a regular basis to make and check conjectures and to visualize the geometric objects I work with.
%\vspace{4mm}

%Taking these factors into account, I believe that the participation in this summer school would be beneficial to my scientific outlook. Therefore I would like to apply for financial support to cover my stay at CIRM.
%\vspace{4mm}

%Thank you for your consideration.
%\vspace{4mm}

%Sincerely,

%Ivan Yakovlev

%\newpage

\makecvtitle % Print the CV title

%----------------------------------------------------------------------------------------
%	EDUCATION SECTION
%----------------------------------------------------------------------------------------

\vspace{-6mm}

\section{Education}
\cventry{2021--2024}{PhD in Pure Mathematics}{Université de Bordeaux}{}{}{At Laboratoire Bordelais de Recherche en Informatique (LaBRI).\\
Combinatorics and Interactions group.\\
Thesis advisor: Vincent Delecroix. Thesis title: Quadrangulations with monodromy constraints.
}
\cventry{2018--2021}{Normalien at École Normale Supérieure de Paris}{}{}{}{Admitted via the International Selection program. 3 year scholarship.}
\cventry{2019--2021}{Master of Mathematics,  Fundamental Mathematics track}{Sorbonne Université}{}{}{}
\cventry{2018--2019}{Bachelor (Licence) in Mathematics}{Sorbonne Université}{}{}{}
\cventry{2015--2019}{Bachelor in Applied Mathematics}{Taras Shevchenko National University of Kyiv}{}{}{}
%\cventry{Mai 2015}{Éducation Secondaire Générale Complète}{Kyievo-Pecherskyi Lycée No.171 ''Leader''}{Kyiv, Ukraine}{}{}




%\section{Mémoire de M2}

%\cvitem{Titre}{\emph{Contribution of maximal-cylinder surfaces to the Masur-Veech volumes of the minimal strata of Abelian differentials.}}
%\cvitem{Directeur}{Vincent Delecroix, \emph{LaBRI, Université de Bordeaux}.}
%\cvitem{Note}{20/20}

%----------------------------------------------------------------------------------------
%	WORK EXPERIENCE SECTION
%---------------------------------------------------------------------------------------

\section{Research interests}
\cvitem{}{Enumeration of square-tiled surfaces according to their number of cylinders.\newline
In general: moduli spaces of curves/differentials (hyperbolic/flat surfaces), combinatorial maps (ribbon graphs), Hurwitz theory, intersection theory on moduli spaces.}

\section{Publications}
\cvitem{}{\small{\textbf{Contribution of $n$-cylinder square-tiled surfaces to Masur--Veech volume of $\mathcal{H}(2g-2)$},
\newline Geometric and Functional Analysis (GAFA), 2023.
\newline DOI link: \url{https://doi.org/10.1007/s00039-023-00652-9}
\newline Arxiv link: \url{https://arxiv.org/abs/2209.12348}}
}

\section{Refereed proceedings}
\cvitem{}{\small{\textbf{Algorithms for length spectra of combinatorial tori} (extended abstract),\newline
\emph{with V. Delecroix, M. Ebbens, F. Lazarus.}
\newline 39th International Symposium on Computational Geometry (SoCG 2023),\newline Vol. 258 of Leibniz International Proceedings in Informatics (LIPIcs), pages 26:1–26:16, Dagstuhl, Germany, 2023.
\newline DOI link: \url{https://doi.org/10.4230/LIPIcs.SoCG.2023.26}}
\newline Full version of this paper was accepted to the \emph{Journal of Computational Geometry}.
\newline Arxiv link: \url{https://arxiv.org/abs/2303.08036}
}

\section{Talks}
\cvitem{Jan 2024}{Journ\'{e}es de combinatoire de Bordeaux, \emph{LaBRI, Bordeaux}.}
\cvitem{Oct 2023}{Geometry and Topology seminar, \emph{Universit\'{e} du Luxembourg}.}
\cvitem{May 2023}{LAMFA PhD Seminar, \emph{LAMFA, Amiens}.}
\cvitem{Mar 2023}{ALEA days, \emph{CIRM, Marseille}.}
\cvitem{Mar 2023}{Journée cartes at Marne-la-Valée, \emph{Université Gustave Eiffel, Marne}.}
\cvitem{Jan 2023}{Seminar of Combinatorics and Interactions group, \emph{LaBRI, Bordeaux}.}
\cvitem{Nov 2022}{Algebraic geometry and moduli seminar, \emph{ETH Zurich}.}
\cvitem{Apr 2022}{Workshop of ANR MoDiff, \emph{LaBRI, Bordeaux}.}
\cvitem{Jan 2022}{Seminar of Combinatorics and Interactions group, \emph{LaBRI, Bordeaux}.}

\section{Visits}
\cvitem{Oct 2023}{Universit\'{e} du Luxembourg (1 week).}
\cvitem{May 2023}{Institut Fourier, Grenoble (1 week).}
\cvitem{Nov 2022}{ETH Zurich, (1 week).}
\cvitem{May 2021}{M2 internship visit, LaBRI, Bordeaux, (1 month).}

\section{Teaching}
\cvitem{2023-2024}{Computer networks (2nd year Bachelor).}
\cvitem{}{Introduction to C programming (1st year Bachelor).}
\cvitem{2022-2023}{Algorithms on tree-like data structures (2nd year Bachelor).}
\cvitem{2021-2022}{Combinatorics, Probability, Statistics (2nd year Bachelor).}
\cvitem{}{Algorithms on arrays (1st year Bachelor).}

\section{Organization / outreach}
\cvitem{Mar 2024}{Co-organization of the School on flat surfaces and interactions, Le Teich (Bordeaux).}
\cvitem{2022-2023}{Co-organization of the Combinatorics and Interactions seminar at LaBRI, Bordeaux.}
\cvitem{Jun 2022}{Training of Ukrainian team for international maths olympiads, CIRM, Marseille, France.}
\cvitem{Apr 2019}{Coordinator at European Girls' Mathematical Olympiad, Kyiv, Ukraine.}
\cvitem{2015--2018}{Olympiad mathematics teaching for high-school students, Kyiv, Ukraine.}

\section{Participation in conferences, schools}
\cvitem{Jan 2024}{Sage Days 125, Le Teich (Bordeaux).}
\cvitem{Oct 2023}{Conference ``Probability and Geometry in, on and of non-Euclidean spaces'', CIRM, Marseille.}
\cvitem{Jul 2023}{Research school ``Renormalization and Visualization for packings, billiards and surfaces'', CIRM, Marseille.}
\cvitem{Jul 2023}{IMJ-PRG Summer School 2023 ``Microlocal and probabilistic methods in geometry and dynamics'', Paris.}
\cvitem{Nov 2022}{Meeting of young researchers in geometry and dynamics, Orsay.}
\cvitem{Oct 2022}{School on Random Walks, G\"{o}ttingen.}
\cvitem{Sep 2022}{Conference ``Combinatorics, dynamics and geometry in the moduli spaces'', CIRM, Marseille.}
\cvitem{May 2022}{Conference ``Structures on Surfaces'', CIRM, Marseille.}
\cvitem{Mar 2022}{ALEA days, CIRM, Marseille.}
\cvitem{Jan 2022}{Winter school on billiards, Autrans.}
\cvitem{Sep 2021}{Summer school on topological recursion ``TRSalento'', Otranto.}

\section{Miscellaneous}
\cvitem{}{Languages: English, French, Russian (native), Ukrainian (native).}
\cvitem{}{Programming: C++, Python, SageMath, OCaml.}

\end{document}

